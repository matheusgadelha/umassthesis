
\documentclass[runningheads]{llncs}
\usepackage{graphicx}
\usepackage{times}
\usepackage{epsfig}
\usepackage{amsmath,amssymb} %
\usepackage{hyperref}
\usepackage{color}
\usepackage{paralist}
\usepackage{float}
\usepackage{xspace}

\usepackage{booktabs}
\usepackage{array, makecell, tabularx}
\usepackage{multirow}
\usepackage{wrapfig}
\usepackage{url}

\usepackage[labelfont=bf]{caption}

\usepackage[width=122mm,left=12mm,paperwidth=146mm,height=193mm,top=12mm,paperheight=217mm]{geometry}

\hypersetup{
    colorlinks=true,
    linkcolor=blue,
    filecolor=magenta,      
    urlcolor=cyan,
}

\newcommand{\norm}[1]{\left\lVert#1\right\rVert}

\def\rui#1{\textcolor{blue}{{R: }{#1}}}
\def\matheus#1{\textcolor{red}{{M: }{#1}}}
\def\smaji#1{\textcolor{magenta}{{S: }{#1}}}

\begin{document}
\pagestyle{headings}
\mainmatter
\def\ECCV18SubNumber{196}  %

\title{Multiresolution Tree Networks for\\3D Point Cloud Processing}

\titlerunning{ECCV-18 submission ID \ECCV18SubNumber}

\authorrunning{ECCV-18 submission ID \ECCV18SubNumber}

\author{Matheus Gadelha, Rui Wang and Subhransu Maji\\
University of Massachusetts - Amherst\\
{\tt\small \{mgadelha, ruiwang, smaji\}@cs.umass.edu}
}
\institute{Paper ID \ECCV18SubNumber}


\maketitle

\begin{abstract}
We present multiresolution tree-structured networks to process point clouds for 3D shape understanding and generation tasks. 
Our network represents a 3D shape as a set of locality-preserving 1D ordered list of points at multiple resolutions. 
This allows efficient feed-forward processing through 1D convolutions, coarse-to-fine analysis through a multi-grid architecture, and it leads to faster convergence and small memory footprint during training.
The proposed tree-structured encoders can be used to classify shapes and outperform existing point-based architectures on shape classification benchmarks, while 
tree-structured decoders can be used for generating point clouds directly and they outperform existing approaches for image-to-shape inference tasks learned using the ShapeNet dataset.
Our model also allows unsupervised learning of point-cloud based shapes by using a variational autoencoder, leading to higher-quality generated shapes.
\end{abstract}

\input{symbols.tex}
\unnumberedchapter{Introduction}

The ability of reasoning in a tridimensional space if of paramount importance for any agent in our physical world.
Very early in the evolutionary process, living beings developed mechanisms for sensing the world
in 3D.
Birds, mammals, reptiles and even insects: they all developed some type of \emph{stereopsis} --
the ability to perceive 3D from multiple views.
Since evolution showed us that living beings benefit from 3D reasoning, it makes sense that man-made
agents should be endowed with similar capabilities.
If we want to build machines that can interact with the world around us, grasp objects, avoid obstacles and reason about the space in general, we need to be able build models that allow those machines
to analyze and generate 3D data.
This thesis is focused on developing techniques that allow us to build those models in a variety of scenarios, with a specific focus on deep learning techniques.

We are mainly concerned about two important issues: \emph{representation} and \emph{data}.
Regarding representations, differently from images, it is not clear what is the best way
to represent 3D data in deep learning models.
The first part of this thesis (Chapters 1, 2 and 3) focuses on unstructured representations.
We show how we can build models capable of generating 3D data represented as sets of point clouds
(Chapters 1 and 2) and shape handles (Chapter 3).
Point cloud models have a smaller memory footprint than volumetric and multiview counterparts while
reconstructing more accurate shapes; models based on sets of shape handles have a different goal -- they are designed be
amenable to shape manipulation tasks, like editing and completion.
The second part of this thesis concerns dealing with the lack of 3D data as supervision and
understanding what is the role that neural network architectures play in inducing shape priors.
Chapter 4 describes how to utilize approximate convex decomposition (ACD) as self-supervised learning task
to improve discriminative models of point clouds trained with limited amount of labels.
We show that features learned by computing ACD yield significant improvement in few-shot segmentation
and unsupervised shape classification benchmarks.
Chapters 5 and 6 explore the prior induced by neural networks when generating 3D data.
Chapter 5 studies the case where networks are used to represent manifolds.
We analytically characterize this prior by analyzing the networks' limiting behavior as a Gaussian
Process and show that it yields impressive results in surface reconstruction tasks.
On the other hand, Chapter 6 is focused on reconstructing shapes using volumetric representations
while learning directly from images.
We introduce a series of differentiable projection operators and show applications to
shape reconstruction from silhouettes, depth images and computational tomography.
Chapter 7 builds upon some these operators to tackle a more challenging problems:
lear generative models of 3D data when no 3D or viewpoint information is available.
More preciselly, we present a class of generative adversarial networks, named \prgans, that is
capable of generating 3D shapes from a collection of \emph{unlabeled} images.
Finally, Chapter 8 presents a proposal to leverage some of the techniques developed during
this thesis in more realistic applications.
Instead of using object-oriented approaches as done in the rest of thesis, we plan to investigate
models to generate scenes from single RGB images.
More importantly, we want to train these models using multiple modalities of supervision:
from full 3D information to image level annotations like semantic masks or keypoints.
The next sections will dive into the two main parts of this thesis in more detail.


\section*{Learning from Unstructured Data}

Tridimensional occupancy grids are a natural choice for representing 3D data in deep neural networks.
They are a straightforward extension to raster images and convolutional layers can be seamlessly
applied to this type of data.
Another way to represent 3D data is by simply utilizing multiple 2D images of a 3D object.
We refer to this as multi-view representation.
This type of representation can also be easily integrated with convolutional layers and even
offers the extra advantage of being able to leverage image features pre-trained from massive image
datasets.

\begin{figure}
 \begin{center}
 \includegraphics[width=0.7\linewidth]{figs/shapepartition.png}
 \end{center}
 \vspace{-20pt}
 \caption{\small Point clouds sorted according to spatial partitioning structures induce
 reasonable point correspondence (indicated by similar colors).
 The same structure can also be used to compute multiple point cloud resolutions.
 We build upon these observations to design multi-resolution convolutional operators
 for point cloud data.}
\end{figure}

However, generating 3D shapes poses a more challenging situation.
While generating 3D data, we are primarily concerned about generating surfaces, which are
inherently sparse in the 3D space.
This leads to a big drawback for occupancy grids: models using them require huge amounts of memory,
being prohibitively large when generating high-resolution shapes.
For multi-view representations, there are two main issues:
first, these representations are restricted to representing only the visible portion of the surface --
interior parts are not represented.
Second, it is not clear how to enforce consistency between different views, which leads to a reduced quality in the generated shapes.
Nevertheless, these models are still very memory intensive and do not handle generating
multiple categories of objects.

A reasonable alternative to those 3D representations is utilizing point clouds.
Point clouds are a very compact surface representation -- every point in the point must be part of the surface.
They also naturally support extra surface attributes, like color and normals, and are directly captured by a variety of 3D sensors.
The biggest challenge while using point clouds in deep networks lies in its unstructured nature.
Since they are sets of points, point cloud representations need to be invariant to permutations.
Moreover, differently form multi-view representations and occupancy grids, it is not clear what is the best way to use convolutional layers in point cloud data.

Our attempt in creating generative models for point clouds was bootstrapped by using spatial-partitioning data-structures to assign an approximate correspondence between points of
different point clouds~\cite{pcagan}.
The motivation is simple: if one can induce such correspondence, point clouds can be treated as structured data.
In practical terms, we compute a $kd$-tree for every point cloud and sort the points according to a level-order traversal in the leaves of the tree.
This sorting induces a reasonable correspondence between points, as shown in Figure 1.
Using this correspondence, we compute a linear low-dimensional shape representation that were used to
train the first Generative Adversarial Networks (GANs) for point cloud generation~\cite{pcagan} (\textbf{Chapter 1}).
Later, we noticed that the spatial partitioning induces a local neighborhood that can be successfully
used to define convolutional operations and to represent multiple point cloud resolutions~\cite{mrt} (\textbf{Chapter 2}).
\begin{figure}
\vspace{-20pt}
 \begin{center}
 \includegraphics[width=0.7\linewidth]{figs/mrtresults.png}
 \end{center}
 \vspace{-20pt}
 \caption{\small Single-view reconstruction using MRTNets.}
 \vspace{-5pt}
\end{figure}
We called these models Multi-Resolution Tree Networks (MRTNets) and applied them to a variety of
discriminative and generative tasks, like shape classification, 
part segmentation, single-view reconstruction and VAEs.
Some of the results are presented in Figure 2.
The models have a small memory footprint when compared against multi-view and occupancy grids counterparts while yielding state-of-the-art results for point cloud classification, single-view reconstruction and
unsupervised feature learning benchmarks.

However manipulating shapes represented as point clouds is a complicated task.
Suppose one wants to edit the wings of an airplane and make them a bit larger.
If the airplane is represented as a set of points that means manually selecting and displacing
a big number of elements, which is a very laborious, borderline unfeasible task.
To this end, multiple techniques have been developed to summarize 3D shapes a set of
simpler shapes that are amenable to manipulations.
We refer to those as \emph{shape handles}.
Thus, we build upon our previous point cloud work and develop generative models capable
of creating shapes represented as sets of shape handles~\cite{shapehandles} (\textbf{Chapter 3}).
We show how those models can be trained an utilized in applications like
shape editing, creation, parsing and interpolation.


\section*{Learning with Limited 3D Supervision and Shape Priors}

Gathering 3D data is a laborious task.
Popular 3D shape benchmarks are orders of magnitude smaller than their image counterparts.
In this context, building models and training strategies that rely in less training data is specially
important.
In this thesis, we tackle this problem in multiple manners.

We start by investigating discriminative models for 3D data trained with limited amount of labels.
Simply gathering 3D data is by itself a problem, but labeling such data is equally problematic.
While many methods have been developed to improve the efficiency of labeling tasks for 3D shapes,
it is still highly desirable to develop label-efficient models that can leverage data from vast
unlabelled shape repositories.
To this end, we propose to utilize Approximate Convex Decomposition (ACD) as a self-supervised task for
training neural networks (\textbf{Chapter 4}).
We show how point cloud architectures can learn ACD by posing it as a metric learning problem trained using automatically computed labels from raw shape representations (meshes, volumes or point clouds). 
Our experiments indicate that multiple neural networks architectures trained in this fashion achieve
state of the art results in few shot part segmentation and unsupervised shape classification benchmarks.

Another way to tackle the lack of 3D data available is to investigate the role of neural
network architectures as priors for shape generation.
For images, recent work~\cite{dip} has showed that convolutional architectures induce natural image priors
that allow them to be used for multiple reconstruction tasks without requiring any training data.
We investigate an analogous behavior for two types of 3D representations in different contexts.
We start by describing how some popular neural network architectures for shape generation can be
posed as manifold parametrizations and induce useful priors for manifold reconstruction
(\textbf{Chapter 5}).
We further analyze the limiting behavior of those models as Gaussian Processes (GPs) and
analytically characterize such prior by deriving its kernel.
We also develop regularization techniques that further improve shape reconstructions in setups
with and without training data.
In the following chapter, we use convolutional architectures to generate volumetric shape
representations coupled with differentiable projection operators to reconstruct shapes from
a set of images~\cite{deepshapeprior} (\textbf{Chapter 6}).
We show how volumetric priors induced by those convolutional architectures can be used in applications
like shape reconstruction from silhouettes, depth maps, and computational tomography.

Finally, we investigate how to utilize image supervision to train 3D generative models.
As mentioned before, image data is considerably more prevalent than 3D.
For example, the most used shape classification benchmark, ModelNet40, contains about 10 thousand shapes,
whereas the most popular image classification benchmark, ImageNet, contains about 14 million images.
Nevertheless, images usually contain real world entities which are inherently 3D.
In other words, a lot of 3D information is encoded in images and being able to leverage that information
to learn to generate 3D shapes is key to build models that can overcome the lack of 3D training data.
We study this issue within a very challenging problem setup.
Consider a set of silhouette images like the ones in Figure 3.
Those images represent silhouettes of various objects from the same category.
If we have viewpoint annotation and object identification (i.e. which images correspond to the same object) this problem can be easily solved using visual hull, which corresponds to the setup
analyzed in Chapter 6.
We can make the problem harder by assuming that no viewpoint annotation is available.
In that case, we can probably achieve a reasonable result by relying in Structure from Motion (SfM) techniques.
The most difficult setup occurs when we neither have object identification nor viewpoint annotation.
In this scenario, one can only rely on non-rigid SfM, which require a strong prior over the generated shapes.
What happens when we have no information regarding 3D shapes? Can we still do something about it?


\begin{figure}
\vspace{-20pt}
 \begin{center}
 \includegraphics[width=0.8\linewidth]{figs/prganpp.png}
 \end{center}
 \vspace{-20pt}
 \caption{\small \prgan is capable of learning generative models of 3D data without using any 3D supervision.
 The core of the approach is the utilization of differentiable projection operators
 that turn 3D representation into images of silhouettes and segmentation masks.}
 \vspace{-5pt}
\end{figure}
Our solution consists of utilizing deep generative models coupled with some of the differentiable projection operators described in Chapter 4.
The intuition is simple: given a dataset with images, we want to generate 3D shapes that, when projected into the image plane, will look like they came from the dataset.
More precisely, we want to match the distribution of images in the dataset to the images created by projecting the generated 3D shapes.
Fortunately, there is a class of deep learning models which is remarkably good in mimicking target image
distributions: generative adversarial networks (GANs).
Thus, we augment the GAN generator with a differentiable shape projection module which turns 3D shapes into silhouette images.
The result is a 3D generative model that is trained without ever seeing any 3D data, only silhouettes of 3D objects (\textbf{Chapter 7}).
We name this model Projective GAN (\prgan)~\cite{prgan,prganijcv}.
Additionally, we extended the projection operators from Chapter 6 to enable learning from extra image annotations 
allowing training \prgans from
part-segmented images~\cite{prganijcv}.

\vspace{-12pt}
\section{Related Work}\label{sec:related}

\noindent \textbf{Generative models for 3D shapes.} Recently, Wu~\etal in~\cite{wu2016learning} proposed a generative model of 3D shapes represented by voxels, using a variant of GAN adapted to 3D convolutions. Two other works are also related. Rezende~\etal~\cite{rezende2016unsupervised} show results for 3D shape completion for simple shapes when views are provided, but require the viewpoints to be known and the generative models are trained on 3D data. Yan~\etal in~\cite{yan2016perspective} learn a mapping from an image to 3D using multiple projections of the 3D shape from known viewpoints (similar to a visual-hull
technique). However, these models operate on a voxel representation of 3D shape, which is difficult to scale up to higher resolution. The network also contains a large number of parameters, which are difficult and take a long time to train. Our method uses spatially partitioned point cloud to represent each shape. It is considerably more lightweight and easy to scale up. In addition, by using a linear shape basis, our network is small hence much easier and faster to train. Through experiments we show that the benefits of this lightweight approach come with no loss of quality compared to previous work. Several recent techniques~\cite{tatarchenko2017octree,Riegler2017CVPR} have explored multi-resolution voxel representations such as \emph{octrees}~\cite{meagher1982geometric} to improve their memory footprint at the expense of additional book keeping. But it remains unclear if 3D-GANs can generate high-resolution sparse outputs.

\vspace{8pt}
\noindent \textbf{Learning a 3D shape bases using point-to-point correspondence.} Another line of work aims to learn a shape basis from data assuming a global alignment of point clouds across models. 
Blanz and Vetter in~\cite{blanz1999morphable} popularized the 3D morphable models for faces which are learned by a PCA analysis of the point clouds across a set of faces with known correspondences. 
The same idea has also been applied to human bodies~\cite{allen2003space}, and other deformable categories~\cite{kar2015category}. 
However, establishing the point-to-point correspondence between 3D shapes is a challenging problem. 
Techniques are based on global rigid or non-rigid pairwise alignment (\eg, \cite{besl1992method,chen1992object,bronstein2010gromov}), learning feature descriptors for matching (\eg, techniques in this survey~\cite{van2011survey}), or fitting a parametric model to each instance (\eg,~\cite{cashman2013shape,prasad2010finding}).
Some techniques improve pairwise correspondence by enforcing cycle-consistency across instances~\cite{huang2013consistent}. 
However, none of these techniques provide consistent global correspondences for shapes with varying and complex structures (\eg, chairs and airplanes). 
Our method uses spatial sorting based on a kd-tree structure. It is a fast and lightweight approximation to the correspondence problem. However, unlike alignment-based approaches, one drawback of the kd-tree sorting is that it's not robust to rotations of the model instances. This is also a drawback of the voxel-based representations.
The ShapeNet dataset~\cite{chang2015shapenet} used in our experiments already contains objects that are consistently oriented, but otherwise one can apply automatic techniques (\eg,~\cite{Su_2015_ICCV}) for viewpoint estimation to achieve this.

\vspace{8pt}
\noindent \textbf{Representing shapes as sets.} Another direction is to use loss functions over \emph{sets} such as Chamfer, Hausdroff, or Earth Mover's Distance (EMD) to estimate similarity between models. The recent work of Fan~\etal~\cite{fan2016point} explores this direction and trains a neural network to generate points from a single image by minimizing the EMD between the generated points and the model points. To apply this approach for shape generation one requires the evaluation of the loss of a generated shape with respect to a distribution of target shapes. While this can be approximated by computing statistics of EMD distance of the generated shape to all shapes in the training data, this is highly inefficient since EMD computation alone scales cubically with the number of points. 
Thus training neural architectures to generate and evaluate loss functions over sets efficiently remains an open problem. 
The approximate ordering induced by the kd-tree allows efficient matrix operations on the \emph{ordered} vector of point coordinates for training shape generators and discriminators.

\section{Method}\label{sec:method}

\begin{figure}
\centering
\includegraphics[width=0.49\linewidth]{PCAGAN/images/airplanes_sorting_new2.png}
\includegraphics[height=1.4in]{PCAGAN/images/vline.png}
\includegraphics[width=0.49\linewidth]{PCAGAN/images/chairs_sorting.png}
\vspace{-12pt}
\caption{\small \label{fig:point_sorting} Visualization of spatially partitioned points for six training shapes from each category. Every point is colored by its index in the sorted order. This shows that the kd-tree sorting leads to reasonably good correspondences between points across all shapes.}
\vspace{-12pt} 
\end{figure}

This section explains our method. To begin, we sample each training 3D shape using Poisson Disk sampling~\cite{Bowers:2010:PPD} to generate a consistent number of evenly distributed points per shape. We typically sample each shape with 1K points, and this can be easily increased or decreased based on actual need. We then build a kd-tree data structure for each point cloud to spatially partition the points and order them consistently across all shapes. Next, we compute the PCA bases using all the point data. 
%The ordering of points in each shape can be further refined by iteratively swapping points to minimize the PCA reconstruction error.
Finally, we train a GAN on the shape coefficients to learn the multi-modal distribution of these coefficients and use it to generate new shapes.

\vspace{4pt}
\noindent \textbf{Spatially partitioned point cloud.} We use $\{\point_i^s\}$ to represent a point cloud where $i$ is the point index and $s$ is the shape index. By default the point data $\point$ includes the $x,y,z$ coordinates of a point, but can include additional attributes such as normal and color etc. We assume each point cloud is centered at the origin and the bounding box is normalized so that the longest dimension spans [-0.5, 0.5]. For each point cloud we build a kd-tree by the following procedure: we start by sorting the entire point cloud along the $x$-axis, and split it in half, resulting in a left subset and a right subset; we then recursively split each of the two subsets, but this time along the $y$-axis; then along $z$-axis, and so on. Basically it's a recursive splitting process where the splitting axis alternates between $x$, $y$, and $z$. The splitting axes can also be chosen in other ways (such as using the longest dimension at each split) to optimize the kd-tree building, but it needs to be consistent across all point clouds.

The kd-tree building naturally sorts the point cloud spatially, and is consistent across all shapes. For example, if we pick the first point from each sorted point cloud, they all have the same spatial relationship to the rest of the points. As a result, this establishes reasonably good correspondences among the point clouds. Figure~\ref{fig:point_sorting} shows an illustration.

%\vspace{4pt}
\noindent \textbf{Computing PCA bases.} We use PCA analysis to derive a linear shape basis on the spatially partitioned point clouds. To begin, we construct a matrix $\matrix$ that consists of the concatenated $x,y,z$ coordinates of each point cloud and all shapes in a given category. The dimensionality of the matrix is $3\,\npoints\times\nshapes$ where $\npoints$ is the number of points in each shape, and $\nshapes$ is the number of shapes. We then perform a PCA on the matrix: $\matrix=U \Sigma V$, resulting in a linear shape basis $U$. Thanks to point sorting using kd-tree, a small basis set is sufficient to well represent all shapes in a category. We use $\nbasis$ to represent the size of the shape basis, and by default choose $\nbasis=100$, which has worked well for all ShapeNet categories we experimented with. The choice of $\nbasis$ can be observed from the rapid dropping of singular values $\Sigma$ following the PCA analysis. Without a good spatial sorting method, it would require a significantly larger basis set to accurately represent all shapes.

To include other point attributes, such as normal, we can concatenate these attributes with the $x,y,z$ coordinates. For example, a matrix that consists of both point and normal data would be $6\,\npoints\times\nshapes$ in size. We suitable increase the basis size (e.g. by choosing $\nbasis=200$) to accommodate the additional data. The rest of the PCA analysis is performed the same way.

%\paragraph{Optimizing point ordering.} While sorting using the kd-tree creates good initial correspondences between points one can further optimize the point ordering by iteratively reducing the PCA reconstruction error.

%\vspace{4pt}
\noindent \textbf{Learning shape coefficients using GAN.} Our method employs a GAN to learn the distribution over the shape coefficients. % on the PCA basis computed in the above step.
Following the PCA analysis step, the matrix $V$ captures the coefficients for all training shapes, i.e. the projections of each point cloud onto the PCA basis. It provides a compact and yet accurate approximation of the 3D shapes. Therefore our generative model only needs to learn to generate the shape coefficients. 
%we can project the point clouds in our training data into this basis and have a compact representation for each one of our training samples.
%In other words, instead of learning how to generate a complete point cloud, our model only has to learn how to generate the coefficients obtained by projecting the point cloud in a linear basis.
Since the dimensionality of the shape basis ($\nbasis=100$) is much smaller (than the number of points on each shape), we can train a GAN to learn the distribution of coeffcients using
a simple and lightweight architecture. In our setup, the random encoding $z$ is a 100-D vector. The generator and discriminator are both fully connected neural networks consisting of 4 layers each, with 100 nodes in each layer.
Each layer is followed by a batch normalization step.
Following the guidelines of previous architectures \cite{wu2016learning}, our discriminator
uses a LeakyReLU activation while our generator uses regular ReLU.



The discriminator is trained by minimizing the vanilla GAN loss described as follows:
\begin{equation}\label{eqn:gan}
	\mathcal{L}_d = \mathbb{E}_{x\sim{\cal T}} [ \log \left(D(x)\right) ] + \mathbb{E}_{z\sim U} [ \log \left(1-D(G(z))\right) ].
\end{equation}
where $x$ represents the shape coefficients, $D$ is the discriminator, $G$ is the generator, $U$ represents an uniform distribution of real numbers in $(-1, 1)$,
and $\mathcal{T}$ is the training data.
In our experiments, we noticed that using the traditional loss for the generator leads to a highly
unstable training where the generated data converges to a single mode (which loses diversity).
To overcome this issue, we employ an approach similar to the one proposed in \cite{improvedGAN}.
Specifically, let $f(x)$ be the intermediate activations of the discriminator given an input $x$.
Our generator will try to generate samples that match some statistics of the activations of
the real data, namely mean and covariance.
Thus, the generator loss is defined as follows:
\begin{equation}\label{eqn:generator}
	\mathcal{L}_g = \norm{\mathbb{E}_{x\sim{\cal T}} [ f(x) ] - \mathbb{E}_{z\sim U} [ f(G(z)) ]}_2^2 +
					\norm{cov_{x\sim{\cal T}} [ f(x) ] - cov_{z\sim U} [ f(G(z)) ]}_2^2
\end{equation}
where $cov$ is the vectorized covariance matrix of the activations.
Using this loss results in a much more stable learning procedure.
During all our experiments the single mode problem never occurred, even when
training the GAN for thousands of epochs.
We use the Adam optimizer~\cite{Adam} with a learning rate of $10^{-4}$ for the discriminator and $0.0025$ for the
generator.
Similarly to \cite{wu2016learning}, we only train the discriminator if its accuracy is below 80\%.

\begin{figure}[t]
\includegraphics[width=1.0\linewidth]{PCAGAN/images/gallery/airplanes.png}
\includegraphics[width=1.0\linewidth]{PCAGAN/images/hline.png}
\includegraphics[width=1.0\linewidth]{PCAGAN/images/gallery/chairs.png}
\includegraphics[width=1.0\linewidth]{PCAGAN/images/hline.png}
\includegraphics[width=1.0\linewidth]{PCAGAN/images/gallery/cars.png}
\vspace{-16pt}
\caption{\small \label{fig:gallery} A gallery showing results of using our method to generate points clouds for three categories: airplane, chair, and car. We use our method to train a GAN for each category separately. The training is generally very fast and completes within a few minutes. The results shown here are generated by randomly sampling the encoding $z$ of the GAN.}
\vspace{-12pt}
\end{figure}




\section{Experimental Results and Discussions}
This section presents experimental results. We implemented \mrtnet using PyTorch. 

\subsection{Shape classification} To demonstrate the effectiveness of the multiresolution encoder, we trained a baseline model that follows the same classification model but replacing multiresolution convolutions with single-scale 1D convolutions. Also, we apply the same test-time data augmentation and compute the test-time average as described in the Section~\ref{mrt:method}. %

\begin{table}[t]
    \begin{minipage}{0.5\linewidth}    
    \centering
    \begin{tabular}{p{4cm}>{\centering\arraybackslash}p{1.2cm}}
        \toprule
        Method & Accuracy\\
        \midrule
        \multicolumn{2}{l}{\textit{View-based methods}}  \\
        MVCNN~\cite{mvcnn}              &  90.1   \\
        MVCNN-MultiRes~\cite{qi2016volumetric}     &  91.4    \\
        \midrule
        \multicolumn{2}{l}{\textit{Point-based methods (w/o normals)}}   \\
        KDNet (1K pts)~\cite{Klokov_2017_ICCV}  & 90.6 \\
        PointNet (1K pts)~\cite{pointnet}   & 89.2 \\
        PointNet++ (1K pts)~\cite{pointnet2} &  90.7 \\
        MRTNet (1K pts)  & \textbf{91.2} \\
        MRTNet (4K pts)& \textbf{91.7} \\
        KDNet (32K pts)~\cite{Klokov_2017_ICCV}     & \textbf{91.8} \\
        \midrule
        \multicolumn{2}{l}{\textit{Point-based methods (with normals)}}   \\
        PointNet++ (5K pts)~\cite{pointnet2} &  91.9\\
        \midrule
        \multicolumn{2}{l}{\textit{Voxel-based methods}}   \\
        OctNet~\cite{Riegler2017CVPR}       & 86.5 \\
        O-CNN~\cite{ocnn}   & 90.6\\
        \midrule
   \end{tabular}
   \caption*{\small (a) \textbf{Comparisons with previous work}. Among point-based methods that use $xyz$ data only, ours is the best in the 1K points group; and our 4K result is comparable with KDNet at 32K points.}
   \end{minipage}
   \begin{minipage}{0.5\linewidth}
   \centering
    \begin{tabular}{p{4.0cm}>{\centering\arraybackslash}p{1.2cm}}
        \toprule
        Method & Accuracy\\
        \midrule
        Full model (MRTNet, 4K pts) & 91.7 \\
        Filters/4 & 91.7 \\
        Single res. & 89.3 \\
        Single res., no aug. (kd-tree) & 86.2 \\
        Single res., no aug. (rp-tree) & 87.4\\
        \midrule
        \end{tabular}
        \caption*{\small (b) \textbf{\mrtnet ablation studies}. Filters/4 reduces the number of filters in each layer by 4. The last three rows are the single resolution model.}

    \begin{tabular}{p{4.0cm}>{\centering\arraybackslash}p{1.2cm}}
        \midrule
        Method & Accuracy\\
        \midrule
        SPH~\cite{Kazhdan:2003} & 68.2 \\
        LFD~\cite{Chen03} & 75.5 \\
        T-L Network\cite{Girdhar16} & 74.4 \\
        VConv-DAE \cite{Sharma2016} & 75.5 \\
        3D-GAN~\cite{3dgan} & 83.3 \\
        MRTNet-VAE (Ours) & \textbf{86.4} \\
        \midrule
        \caption*{\small (c) \textbf{Unsupervised representation learning}. Section~\ref{sec:exp_gen}.}
        \end{tabular}
   \end{minipage}
    \caption{Instance classification accuracy on the ModelNet40 dataset.} \label{tab:class}
    \vspace{-18pt}
\end{table}

Classification benchmark results are in Table~\ref{tab:class}(a). 
As shown in the table, \mrtnet achieves the best results among all \textbf{point-based} methods that use $xyz$ data only. In particular, ours is the best in the 1K points group. We also experimented with sampling shapes using 4K points, and the result is comparable with KDNet at 32K points -- in this case, KDNet uses 8$\times$ more points (hence 8$\times$ more memory) than ours, and is only $0.1\%$ better. PointNet++~\cite{pointnet2} with 5K points and normals is $0.2\%$ better than ours.

\setlength{\intextsep}{0pt}%
\setlength{\columnsep}{0pt}%
\begin{wrapfigure}{r}{0.35\linewidth} 
\includegraphics[width=1.0\linewidth]{MRTNet/imgs/convergence_loss.pdf}
\vspace{-20pt}
\caption{\small Cross entropy decay \label{fig:convergence}}
\end{wrapfigure} 
Table~\ref{tab:class}(b) shows ablation study results with variants of our approach.
Particularly, the multiresolution version is more than $2\%$ better than the baseline model (i.e. single resolution), 
while using the same number of parameters (the Filters/4 version). 
Besides, \mrtnet converges must faster than the baseline model, as we can see in the cross entropy loss decay plots in Figure~\ref{fig:convergence}. 
This shows that the multiresolution architecture leads to higher quality/accuracy and is memory efficient. 

Our single resolution baseline is akin to KDNet except it doesn't condition the convolutions on the splitting axes.
It results in $1.3\%$ less classification accuracy compared to KDNet (1K pts). This suggests that conditioning on the splitting axes during convolutions improves the accuracy.
However, this comes at the cost of extra book keeping and at least three times more parameters.
\mrtnet achieves greater benefits with lesser overhead.
Similar to the KDNet, our methods also benefit from data augmentation and can be used with both kd-trees and rp-trees.






\begin{table}[t]
\scriptsize
\centering
\begin{tabular}{c||c|c|c|c|c|c}
\hline
\multicolumn{1}{c||}{\multirow{2}{*}{\bf Category}} & \multicolumn{3}{c|}{3D-R2N2~\cite{choy20163d}} & Fan et al.~\cite{fan2016point} & Lin et al.~\cite{lin2018learning} & \mrtnet \\
& 1 view    & 3 views & 5 views & (1 view) & (1 view) & (1 view) \\ \hline
airplane & 3.207 / 2.879 & 2.521 / 2.468 & 2.399 / 2.391 & 1.301 / 1.488 & 1.294 / 1.541 & {\bf 0.976} / {\bf 0.920}\\
bench & 3.350 / 3.697 & 2.465 / 2.746 & 2.323 / 2.603 & 1.814 / 1.983 & 1.757 / 1.487 & {\bf 1.438} / {\bf 1.326}\\
cabinet & 1.636 / 2.817 & 1.445 / 2.626 & {\bf 1.420} / 2.619 & 2.463 / 2.444 & 1.814 / {\bf 1.072} & 1.774 / 1.602\\
car & 1.808 / 3.238 & 1.685 / 3.151 & 1.664 / 3.146 & 1.800 / 2.053 & 1.446 / {\bf 1.061} & {\bf 1.395} / 1.303\\
chair & 2.759 / 4.207 & 1.960 / 3.238 & 1.854 / 3.080 & 1.887 / 2.355 & 1.886 / 2.041 & {\bf 1.650} / {\bf 1.603}\\
display & 3.235 / 4.283 & 2.262 / 3.151 & 2.088 / 2.953 & 1.919 / 2.334 & 2.142 / {\bf 1.440} & {\bf 1.815} / 1.901\\
lamp & 8.400 / 9.722 & 6.001 / 7.755 & 5.698 / 7.331 & 2.347 / 2.212 & 2.635 / 4.459 & {\bf 1.944} / {\bf 2.089}\\
speaker & 2.652 / 4.335 & 2.577 / 4.302 & 2.487 / 4.203 & 3.215 / 2.788 & 2.371 / {\bf 1.706} & {\bf 2.165} / 2.121\\
rifle & 4.798 / 2.996 & 4.307 / 2.546 & 4.193 / 2.447 & 1.316 / 1.358 & 1.289 / 1.510 & {\bf 1.029} / {\bf 1.028}\\
sofa & 2.725 / 3.628 & 2.371 / 3.252 & 2.306 / 3.196 & 2.592 / 2.784 & 1.917 / {\bf 1.423} & {\bf 1.768} / 1.756\\
table & 3.118 / 4.208 & 2.268 / 3.277 & 2.128 / 3.134 & 1.874 / 2.229 & 1.689 / 1.620 & {\bf 1.570} / {\bf 1.405}\\
telephone & 2.202 / 3.314 & 1.969 / 2.834 & 1.874 / 2.734 & 1.516 / 1.989 & 1.939 / {\bf 1.198} & {\bf 1.346} / 1.332\\
watercraft & 3.592 / 4.007 & 3.299 / 3.698 & 3.210 / 3.614 & 1.715 / 1.877 & 1.813 / 1.550 & {\bf 1.394} / {\bf 1.490}\\
\hline
 \bf mean& 3.345 / 4.102 & 2.702 / 3.465 & 2.588 / 3.342 & 1.982 / 2.146 & 1.846 / 1.701 & {\bf 1.559} / {\bf 1.529}\\
\hline
\end{tabular}
\caption{\label{table:multi} \small \textbf{Single-image shape inference results}. The training data consists of 13 categories of shapes provided by~\cite{choy20163d}.     The numbers shown are [pred$\to$GT / GT$\to$pred] errors, scaled by 100. The mean is computed across all 13 categories. Our \mrtnet produces 4K points for each shape.
}
\vspace{-12pt}
\end{table}

\begin{table}[t]
\centering
\begin{tabular}{|c|c|c|}
\hline
Fully Connected & Single Res. & MRTNet \\
1.824 / 2.297 & 1.708 / 1.831 & {\bf 1.559} / {\bf 1.529} \\
\hline
\end{tabular}
\caption{\small
\label{table:ablation}
\textbf{Ablation studies for the image to shape decoder.} The numbers shown are [pred$\to$GT / GT$\to$pred] errors, scaled by 100. 
The values are the mean computed across all 13 categories.}
\vspace{-24pt}
\end{table}

\subsection{Single-image shape inference} \label{sec:exp_shapeinfer} We compare our single-image shape inference results with volumetric~\cite{choy20163d}, view-based~\cite{lin2018learning} and point-based~\cite{fan2016point} approaches using the evaluation metric by~\cite{lin2018learning}. 
Given a source point cloud $\mathbf{x}$ and a target point cloud $\mathbf{y}$, we compute
the average euclidean distance from each point in $\mathbf{x}$ to its closest in $\mathbf{y}$.
We refer to this as pred$\to$GT (prediction to groundtruth) error. It indicates how dissimilar the predicted shape is from the ground-truth.
The GT$\to$pred error is computed similarly by swapping $\mathbf{x}$ and $\mathbf{y}$, and it measures coverage (i.e. how complete the ground-truth surface was covered by the prediction).
For the voxel based model~\cite{choy20163d}, we used the same procedure as~\cite{lin2018learning},
where point clouds are formed by creating one point in the center of each surface voxel.
Surface voxels are extracted by subtracting the prediction by
its eroded version and rescale them such that the tightest 3D bounding boxes of the prediction and
the ground-truth CAD models have the same volume.

\begin{figure*}[t]
\centering
\setlength{\tabcolsep}{0pt}
\begin{tabular}{ccc|ccc|ccc}
\includegraphics[width=.108\linewidth]{MRTNet/rendering/srfc_comparison/FCI2PC_all_vgg_True/cc25ba35b3f6e8d3d064b65ccd8977_mrt_v1.png} &
\includegraphics[width=.108\linewidth]{MRTNet/rendering/srfc_comparison/FCI2PC_all_vgg_True/cc2930e7ceb24691febad4f49b26ec_mrt_v1.png} &
\includegraphics[width=.108\linewidth]{MRTNet/rendering/srfc_comparison/FCI2PC_all_vgg_True/cc03a89a98cd2660c423490470c47d_mrt_v1.png} &
\includegraphics[width=.108\linewidth]{MRTNet/rendering/srfc_comparison/SRI2PC_all_vgg_True/cc25ba35b3f6e8d3d064b65ccd8977_mrt_v1.png} &
\includegraphics[width=.108\linewidth]{MRTNet/rendering/srfc_comparison/SRI2PC_all_vgg_True/cc2930e7ceb24691febad4f49b26ec_mrt_v1.png} &
\includegraphics[width=.108\linewidth]{MRTNet/rendering/srfc_comparison/SRI2PC_all_vgg_True/cc03a89a98cd2660c423490470c47d_mrt_v1.png} &
\includegraphics[width=.108\linewidth]{MRTNet/rendering/srfc_comparison/MRI2PC_all_vgg_True/cc25ba35b3f6e8d3d064b65ccd8977_mrt_v1.png} &
\includegraphics[width=.108\linewidth]{MRTNet/rendering/srfc_comparison/MRI2PC_all_vgg_True/cc2930e7ceb24691febad4f49b26ec_mrt_v1.png} &
\includegraphics[width=.108\linewidth]{MRTNet/rendering/srfc_comparison/MRI2PC_all_vgg_True/cc03a89a98cd2660c423490470c47d_mrt_v1.png} \\

\includegraphics[width=.108\linewidth]{MRTNet/rendering/srfc_comparison/FCI2PC_all_vgg_True/cc25ba35b3f6e8d3d064b65ccd8977_mrt_green_v1.png} &
\includegraphics[width=.108\linewidth]{MRTNet/rendering/srfc_comparison/FCI2PC_all_vgg_True/cc2930e7ceb24691febad4f49b26ec_mrt_green_v1.png} &
\includegraphics[width=.108\linewidth]{MRTNet/rendering/srfc_comparison/FCI2PC_all_vgg_True/cc03a89a98cd2660c423490470c47d_mrt_green_v1.png} &
\includegraphics[width=.108\linewidth]{MRTNet/rendering/srfc_comparison/SRI2PC_all_vgg_True/cc25ba35b3f6e8d3d064b65ccd8977_mrt_green_v1.png} &
\includegraphics[width=.108\linewidth]{MRTNet/rendering/srfc_comparison/SRI2PC_all_vgg_True/cc2930e7ceb24691febad4f49b26ec_mrt_green_v1.png} &
\includegraphics[width=.108\linewidth]{MRTNet/rendering/srfc_comparison/SRI2PC_all_vgg_True/cc03a89a98cd2660c423490470c47d_mrt_green_v1.png} &
\includegraphics[width=.108\linewidth]{MRTNet/rendering/srfc_comparison/MRI2PC_all_vgg_True/cc25ba35b3f6e8d3d064b65ccd8977_mrt_green_v1.png} &
\includegraphics[width=.108\linewidth]{MRTNet/rendering/srfc_comparison/MRI2PC_all_vgg_True/cc2930e7ceb24691febad4f49b26ec_mrt_green_v1.png} &
\includegraphics[width=.108\linewidth]{MRTNet/rendering/srfc_comparison/MRI2PC_all_vgg_True/cc03a89a98cd2660c423490470c47d_mrt_green_v1.png}
\end{tabular}
\vspace{-8pt}
    \caption{\label{fig:ablation-comp} 
    \small Shapes generated by 1) the fully connected baseline; 2) the single-resolution baseline; and 3) \mrtnet.
    Colors in the first row indicate the index of a point in the output point list.}
\vspace{-6pt}
\end{figure*}

\begin{figure*}[t]
\centering
\setlength{\tabcolsep}{0pt}
\begin{tabular}{c|cccccccc}
{\rotatebox[origin=lt]{90}{Input}} &
\includegraphics[width=.12\linewidth]{MRTNet/rendering/i2pc_comparison/c83b3192c338527a2056b4bd5d870b_alpha.png} &
\includegraphics[width=.12\linewidth]{MRTNet/rendering/i2pc_comparison/cbe006da89cca7ffd6bab114dd47e3_alpha.png} &
\includegraphics[width=.12\linewidth]{MRTNet/rendering/i2pc_comparison/cd24768b45ef5efcb1bb46d2556ba6_alpha.png} &
\includegraphics[width=.12\linewidth]{MRTNet/rendering/i2pc_comparison/cdee5ccae3613c507e1dc03b595bd3_alpha.png} &
\includegraphics[width=.12\linewidth]{MRTNet/rendering/i2pc_comparison/d2d645ce6ad43434d42b9650f19dd4_alpha.png} &
\includegraphics[width=.12\linewidth]{MRTNet/rendering/i2pc_comparison/ccc6b5ace9f5164d26068f53fe0ecf_alpha.png} &
\includegraphics[width=.12\linewidth]{MRTNet/rendering/i2pc_comparison/d18592d9615b01bbbc0909d98a1ff2_alpha.png} &
\includegraphics[width=.12\linewidth]{MRTNet/rendering/i2pc_comparison/cceaeed0d8cf5bdbca68d7e2f215cb_alpha.png} \\
\hline
{\rotatebox[origin=lt]{90}{G.T.}} &
\includegraphics[width=.12\linewidth]{MRTNet/rendering/i2pc_comparison/gt/img1.png} &
\includegraphics[width=.12\linewidth]{MRTNet/rendering/i2pc_comparison/gt/img2.png} &
\includegraphics[width=.12\linewidth]{MRTNet/rendering/i2pc_comparison/gt/img3.png} &
\includegraphics[width=.12\linewidth]{MRTNet/rendering/i2pc_comparison/gt/img4.png} &
\includegraphics[width=.12\linewidth]{MRTNet/rendering/i2pc_comparison/gt/img5.png} &
\includegraphics[width=.12\linewidth]{MRTNet/rendering/i2pc_comparison/gt/img6_alt.png} &
\includegraphics[width=.12\linewidth]{MRTNet/rendering/i2pc_comparison/gt/img7.png} &
\includegraphics[width=.12\linewidth]{MRTNet/rendering/i2pc_comparison/gt/img8.png} \\
\hline
{\rotatebox[origin=lt]{90}{\mrtnet}} &
\includegraphics[width=.12\linewidth]{MRTNet/rendering/i2pc_comparison/c83b3192c338527a2056b4bd5d870b_mrt_v1.png} &
\includegraphics[width=.12\linewidth]{MRTNet/rendering/i2pc_comparison/cbe006da89cca7ffd6bab114dd47e3_mrt_v1.png} &
\includegraphics[width=.12\linewidth]{MRTNet/rendering/i2pc_comparison/cd24768b45ef5efcb1bb46d2556ba6_mrt_v1.png} &
\includegraphics[width=.12\linewidth]{MRTNet/rendering/i2pc_comparison/cdee5ccae3613c507e1dc03b595bd3_mrt_v1.png} &
\includegraphics[width=.12\linewidth]{MRTNet/rendering/i2pc_comparison/d2d645ce6ad43434d42b9650f19dd4_mrt_v1.png} &
\includegraphics[width=.12\linewidth]{MRTNet/rendering/i2pc_comparison/ccc6b5ace9f5164d26068f53fe0ecf_mrt_v1.png} &
\includegraphics[width=.12\linewidth]{MRTNet/rendering/i2pc_comparison/d18592d9615b01bbbc0909d98a1ff2_mrt_v1.png} &
\includegraphics[width=.12\linewidth]{MRTNet/rendering/i2pc_comparison/cceaeed0d8cf5bdbca68d7e2f215cb_mrt_v1.png} \\
\hline
{\rotatebox[origin=lt]{90}{\small Fan~\cite{fan2016point}}} &
\includegraphics[width=.12\linewidth]{MRTNet/rendering/i2pc_comparison/c83b3192c338527a2056b4bd5d870b_alignedpsg_v1.png} &
\includegraphics[width=.12\linewidth]{MRTNet/rendering/i2pc_comparison/cbe006da89cca7ffd6bab114dd47e3_alignedpsg_v1.png} &
\includegraphics[width=.12\linewidth]{MRTNet/rendering/i2pc_comparison/cd24768b45ef5efcb1bb46d2556ba6_alignedpsg_v1.png} &
\includegraphics[width=.12\linewidth]{MRTNet/rendering/i2pc_comparison/cdee5ccae3613c507e1dc03b595bd3_alignedpsg_v1.png} &
\includegraphics[width=.12\linewidth]{MRTNet/rendering/i2pc_comparison/d2d645ce6ad43434d42b9650f19dd4_alignedpsg_v1.png} &
\includegraphics[width=.12\linewidth]{MRTNet/rendering/i2pc_comparison/ccc6b5ace9f5164d26068f53fe0ecf_alignedpsg_v1.png} &
\includegraphics[width=.12\linewidth]{MRTNet/rendering/i2pc_comparison/d18592d9615b01bbbc0909d98a1ff2_alignedpsg_v1.png} &
\includegraphics[width=.12\linewidth]{MRTNet/rendering/i2pc_comparison/cceaeed0d8cf5bdbca68d7e2f215cb_alignedpsg_v1.png} \\
\hline
{\rotatebox[origin=lt]{90}{\small Choy~\cite{choy20163d}}} &
\includegraphics[width=.12\linewidth]{MRTNet/rendering/i2pc_comparison/c83b3192c338527a2056b4bd5d870b_r2n2_v1.png} &
\includegraphics[width=.12\linewidth]{MRTNet/rendering/i2pc_comparison/cbe006da89cca7ffd6bab114dd47e3_r2n2_v1.png} &
\includegraphics[width=.12\linewidth]{MRTNet/rendering/i2pc_comparison/cd24768b45ef5efcb1bb46d2556ba6_r2n2_v1.png} &
\includegraphics[width=.12\linewidth]{MRTNet/rendering/i2pc_comparison/cdee5ccae3613c507e1dc03b595bd3_r2n2_v1.png} &
\includegraphics[width=.12\linewidth]{MRTNet/rendering/i2pc_comparison/d2d645ce6ad43434d42b9650f19dd4_r2n2_v1.png} &
\includegraphics[width=.12\linewidth]{MRTNet/rendering/i2pc_comparison/ccc6b5ace9f5164d26068f53fe0ecf_r2n2_v1.png} &
\includegraphics[width=.12\linewidth]{MRTNet/rendering/i2pc_comparison/d18592d9615b01bbbc0909d98a1ff2_r2n2_v1.png} &
\includegraphics[width=.12\linewidth]{MRTNet/rendering/i2pc_comparison/cceaeed0d8cf5bdbca68d7e2f215cb_r2n2_v1.png} \\
\hline
\end{tabular}
\vspace{-8pt}
    \caption{\label{fig:inference-comp} 
    \small Qualitative results for single-image shape inference. From top to bottom: input images, ground truth 3D shapes, results of \mrtnet, Fan et al.~\cite{fan2016point}, and Choy et al.~\cite{choy20163d}.
    }
\vspace{-12pt}
\end{figure*}

Table~\ref{table:multi} shows our results. 
Our solution outperforms competing methods in 12 out of 13 categories on the pred$\to$GT error, and in
6 categories on GT$\to$pred error.
Note that we are consistently better than the point-based methods such as~\cite{fan2016point} in both metrics; 
and we are consistently better than~\cite{lin2018learning} in the pred$\to$GT metric.
Furthermore, our method wins by a considerable margin in terms of the mean per category on both metrics. 
It is important to highlight that the multi-view based method~\cite{lin2018learning} produces tens of thousands of points and many of them
are not in the right positions, which penalizes their pred$\to$GT metric, but that helps to improve their GT$\to$pred.
Moreover, as mentioned in~\cite{lin2018learning}, their method has difficulties capturing thin structures (e.g. lamps) whereas ours is able
to capture them relatively well.
For example, our GT$\to$pred error for the \textbf{lamp} category (which contains many thin geometric structures) is more than two times smaller than the error by~\cite{lin2018learning}, indicating that MRTNet is more successful at capturing thin structures in the shapes.

\para{Ablation studies.}
In order to quantify the effectiveness of the multiresolution decoder, we compared our method with two
different baselines: a fully connected decoder and a single-resolution decoder.
The fully connected decoder consists of 3 linear layers with 4096 hidden neurons, each layer followed by batch normalization 
and ReLU activation units.
On top of that, we add a final layer that outputs $4096\times3$ values corresponding to the final point cloud, followed
by a hyperbolic tangent activation function.
The single resolution decoder follows the same architecture of the MRT decoder but replacing multiresolution convolutions with single-scale 1D convolutions.
Results are shown in Table~\ref{table:ablation}.
Note that both baselines are quite competitive. 
The single-resolution decoder is comparable to the result of~\cite{lin2018learning}, while the fully connected one achieves similar mean errors to~\cite{fan2016point}.
Still, they fall noticeably behind \mrtnet.


\begin{figure*}[t]
\centering
\setlength{\tabcolsep}{0pt}
\begin{tabular}{cccccccc}
    
\includegraphics[width=.12\linewidth]{MRTNet/rendering/real_MRI2PC/out_0000.png} &
\includegraphics[width=.12\linewidth]{MRTNet/rendering/real_MRI2PC/tables/table8_mrt.png} &
\includegraphics[width=.12\linewidth]{MRTNet/rendering/real_MRI2PC/out_0002.png} &
\includegraphics[width=.12\linewidth]{MRTNet/rendering/real_MRI2PC/tables/table2_mrt.png} &
\includegraphics[width=.12\linewidth]{MRTNet/rendering/real_MRI2PC/out_0004.png} &
\includegraphics[width=.12\linewidth]{MRTNet/rendering/real_MRI2PC/tables/table3_mrt.png} &
\includegraphics[width=.12\linewidth]{MRTNet/rendering/real_MRI2PC/tables/table5_mrt.png} &
\includegraphics[width=.12\linewidth]{MRTNet/rendering/real_MRI2PC/out_0007.png} \\

\includegraphics[width=.12\linewidth]{MRTNet/rendering/real_MRI2PC/out_0000_v1.png} &
\includegraphics[width=.12\linewidth]{MRTNet/rendering/real_MRI2PC/tables/table8_mrt_v1.png} &
\includegraphics[width=.12\linewidth]{MRTNet/rendering/real_MRI2PC/out_0002_v1.png} &
\includegraphics[width=.12\linewidth]{MRTNet/rendering/real_MRI2PC/tables/table2_mrt_v1.png} &
\includegraphics[width=.12\linewidth]{MRTNet/rendering/real_MRI2PC/out_0004_v1.png} &
\includegraphics[width=.12\linewidth]{MRTNet/rendering/real_MRI2PC/tables/table3_mrt_v1.png} &
\includegraphics[width=.12\linewidth]{MRTNet/rendering/real_MRI2PC/tables/table5_mrt_v1.png} &
\includegraphics[width=.12\linewidth]{MRTNet/rendering/real_MRI2PC/out_0007_v1.png} \\

\includegraphics[width=.12\linewidth]{MRTNet/rendering/real_MRI2PC/out_0000_v0.png} &
\includegraphics[width=.12\linewidth]{MRTNet/rendering/real_MRI2PC/tables/table8_mrt_v0.png} &
\includegraphics[width=.12\linewidth]{MRTNet/rendering/real_MRI2PC/out_0002_v0.png} &
\includegraphics[width=.12\linewidth]{MRTNet/rendering/real_MRI2PC/tables/table2_mrt_v0.png} &
\includegraphics[width=.12\linewidth]{MRTNet/rendering/real_MRI2PC/out_0004_v0.png} &
\includegraphics[width=.12\linewidth]{MRTNet/rendering/real_MRI2PC/tables/table3_mrt_v0.png} &
\includegraphics[width=.12\linewidth]{MRTNet/rendering/real_MRI2PC/tables/table5_mrt_v0.png} &
\includegraphics[width=.12\linewidth]{MRTNet/rendering/real_MRI2PC/out_0007_v0.png} \\

\hline

\includegraphics[width=.12\linewidth]{MRTNet/rendering/playdoh_shapes/plane3_mrt.png} &
\includegraphics[width=.12\linewidth]{MRTNet/rendering/playdoh_shapes/car8_clipped_rev_1_mrt.png} &
\includegraphics[width=.12\linewidth]{MRTNet/rendering/playdoh_shapes/5701697_clipped_rev_1_mrt.png} &
\includegraphics[width=.12\linewidth]{MRTNet/rendering/playdoh_shapes/cara_clipped_rev_1_mrt.png} &
\includegraphics[width=.12\linewidth]{MRTNet/rendering/playdoh_shapes/ship9_mrt.png} &
\includegraphics[width=.12\linewidth]{MRTNet/rendering/playdoh_shapes/ship3_mrt.png} &
\includegraphics[width=.12\linewidth]{MRTNet/rendering/playdoh_shapes/Play-Doh-Sofa_clipped_rev_1_mrt.png} &
\includegraphics[width=.12\linewidth]{MRTNet/rendering/playdoh_shapes/plane2_mrt.png} \\

\includegraphics[width=.12\linewidth]{MRTNet/rendering/playdoh_shapes/plane3_mrt_v0.png} &
\includegraphics[width=.12\linewidth]{MRTNet/rendering/playdoh_shapes/car8_clipped_rev_1_mrt_v0.png} &
\includegraphics[width=.12\linewidth]{MRTNet/rendering/playdoh_shapes/5701697_clipped_rev_1_mrt_v0.png} &
\includegraphics[width=.12\linewidth]{MRTNet/rendering/playdoh_shapes/cara_clipped_rev_1_mrt_v0.png} &
\includegraphics[width=.12\linewidth]{MRTNet/rendering/playdoh_shapes/ship9_mrt_v0.png} &
\includegraphics[width=.12\linewidth]{MRTNet/rendering/playdoh_shapes/ship3_mrt_v0.png} &
\includegraphics[width=.12\linewidth]{MRTNet/rendering/playdoh_shapes/Play-Doh-Sofa_clipped_rev_1_mrt_v0.png} &
\includegraphics[width=.12\linewidth]{MRTNet/rendering/playdoh_shapes/plane2_mrt_v0.png} \\

\includegraphics[width=.12\linewidth]{MRTNet/rendering/playdoh_shapes/plane3_mrt_v1.png} &
\includegraphics[width=.12\linewidth]{MRTNet/rendering/playdoh_shapes/car8_clipped_rev_1_mrt_v1.png} &
\includegraphics[width=.12\linewidth]{MRTNet/rendering/playdoh_shapes/5701697_clipped_rev_1_mrt_v1.png} &
\includegraphics[width=.12\linewidth]{MRTNet/rendering/playdoh_shapes/cara_clipped_rev_1_mrt_v1.png} &
\includegraphics[width=.12\linewidth]{MRTNet/rendering/playdoh_shapes/ship9_mrt_v1.png} &
\includegraphics[width=.12\linewidth]{MRTNet/rendering/playdoh_shapes/ship3_mrt_v1.png} &
\includegraphics[width=.12\linewidth]{MRTNet/rendering/playdoh_shapes/Play-Doh-Sofa_clipped_rev_1_mrt_v1.png} &
\includegraphics[width=.12\linewidth]{MRTNet/rendering/playdoh_shapes/plane2_mrt_v1.png} \\

\end{tabular}
\vspace{-8pt}
    \caption{\label{fig:real} 
    \small Shapes generated by applying \mrtnet on Inernet photos of furnitures and toys. \mrtnet is trained on the 13 categories of ShapeNet database (Table~\ref{table:multi}) . Note how the network is capable of generating detailed shapes from real photos, even though it is trained only on rendered images using simple shading models. For each output shape we show two different views.
    }
\vspace{-18pt}
\end{figure*}

In Figure~\ref{fig:ablation-comp} we visualize the structures of the output point clouds generated by the three methods.
The point clouds generated by MRTNet
present strong spatial coherence: points that are spatially nearby in 3D are also likely to be nearby in the 1D list.
This coherence is present to some degree in the single-resolution outputs (note the dark blue points in the chair's arms), but is almost completely absent in the results by the fully connected decoder. This is expected,
since fully connected layers do not leverage the spatial correlation of their inputs.
Operating at multiple scales enables MRTNet to enforce a stronger spatial coherence, allowing it to more efficiently synthesize detailed point clouds with coherent geometric structures.

\para{Qualitative results.} In Figure~\ref{fig:inference-comp} we present qualitative results of our method and comparisons to two prior works. 
The input images have 3 color channels and dimensions $224\times224$. 
In Figure~\ref{fig:real} we show results of our method applied on photographs downloaded from the Internet. 
To apply our method, we manually removed the background from the photos using~\cite{clipmagic}, which generally took less than half a minute per photo. 
As seen from the results, MRTNet is able to capture the structure and interesting geometric details of the objects (e.g. wheels of the office chairs), 
even though the input images are considerably different from the rendered ones used in training.


\begin{figure*}[t]
\centering
\setlength{\tabcolsep}{0pt}
\begin{tabular}{ccccccccc}
{\rotatebox[origin=lt]{90}{\mrtnet}} &
\includegraphics[width=.12\linewidth]{MRTNet/rendering/selected/mr_chairs/pc_0001.png} &
\includegraphics[width=.12\linewidth]{MRTNet/rendering/selected/mr_chairs/pc_0006.png} &
\includegraphics[width=.12\linewidth]{MRTNet/rendering/selected/mr_chairs/pc_0007.png} &
\includegraphics[width=.12\linewidth]{MRTNet/rendering/selected/mr_chairs/pc_0014.png} &
\includegraphics[width=.12\linewidth]{MRTNet/rendering/selected/mr_chairs/pc_0027.png} &
\includegraphics[width=.12\linewidth]{MRTNet/rendering/selected/mr_chairs/pc_0036.png} &
\includegraphics[width=.12\linewidth]{MRTNet/rendering/selected/mr_chairs/pc_0045.png} &
\includegraphics[width=.12\linewidth]{MRTNet/rendering/selected/mr_chairs/pc_0063.png} \\
{\rotatebox[origin=lt]{90}{Baseline}} &
\includegraphics[width=.12\linewidth]{MRTNet/rendering/selected/ae_chairs/pc_0000.png} &
\includegraphics[width=.12\linewidth]{MRTNet/rendering/selected/ae_chairs/pc_0001.png} &
\includegraphics[width=.12\linewidth]{MRTNet/rendering/selected/ae_chairs/pc_0003.png} &
\includegraphics[width=.12\linewidth]{MRTNet/rendering/selected/ae_chairs/pc_0004.png} &
\includegraphics[width=.12\linewidth]{MRTNet/rendering/selected/ae_chairs/pc_0006.png} &
\includegraphics[width=.12\linewidth]{MRTNet/rendering/selected/ae_chairs/pc_0007.png} &
\includegraphics[width=.12\linewidth]{MRTNet/rendering/selected/ae_chairs/pc_0008.png} &
\includegraphics[width=.12\linewidth]{MRTNet/rendering/selected/ae_chairs/pc_0009.png} \\
\end{tabular}
\vspace{-8pt}
    \caption{\label{fig:chairs-comp} 
    \small Qualitative comparisons of \mrvae with a single-resolution baseline model. Results are generated by randomly sampling the encoding $\encoding$. 
    \mrvae is able to preserve shape details much better than the baseline model, and produces less noisy outputs.}
\vspace{-6pt}
\end{figure*}
\begin{figure*}[t]
\centering
\setlength{\tabcolsep}{0pt}
\begin{tabular}{cccccccccccccccc}
\includegraphics[width=.1\linewidth]{MRTNet/rendering/selected/rec_shapenet/pc_0000.png} &
\includegraphics[width=.1\linewidth]{MRTNet/rendering/selected/rec_shapenet/pc_0002.png} &
\includegraphics[width=.1\linewidth]{MRTNet/rendering/selected/rec_shapenet/pc_0003.png} &
\includegraphics[width=.1\linewidth]{MRTNet/rendering/selected/rec_shapenet/pc_0004.png} &
\includegraphics[width=.1\linewidth]{MRTNet/rendering/selected/rec_shapenet/pc_0008.png} &
\includegraphics[width=.1\linewidth]{MRTNet/rendering/selected/rec_shapenet/pc_0011.png} &
\includegraphics[width=.1\linewidth]{MRTNet/rendering/selected/rec_shapenet/pc_0013.png} &
\includegraphics[width=.1\linewidth]{MRTNet/rendering/selected/rec_shapenet/pc_0015.png} &
\includegraphics[width=.1\linewidth]{MRTNet/rendering/selected/rec_shapenet/pc_0016.png} &
\includegraphics[width=.1\linewidth]{MRTNet/rendering/selected/rec_shapenet/pc_0018.png} \\
\includegraphics[width=.1\linewidth]{MRTNet/rendering/selected/rec_shapenet/pc_0025.png} &
\includegraphics[width=.1\linewidth]{MRTNet/rendering/selected/rec_shapenet/pc_0026.png} &
\includegraphics[width=.1\linewidth]{MRTNet/rendering/selected/rec_shapenet/pc_0027.png} &
\includegraphics[width=.1\linewidth]{MRTNet/rendering/selected/rec_shapenet/pc_0033.png} &
\includegraphics[width=.1\linewidth]{MRTNet/rendering/selected/rec_shapenet/pc_0035.png} &
\includegraphics[width=.1\linewidth]{MRTNet/rendering/selected/rec_shapenet/pc_0037.png} &
\includegraphics[width=.1\linewidth]{MRTNet/rendering/selected/rec_shapenet/pc_0042.png} &
\includegraphics[width=.1\linewidth]{MRTNet/rendering/selected/rec_shapenet/pc_0044.png} &
\includegraphics[width=.1\linewidth]{MRTNet/rendering/selected/rec_shapenet/pc_0048.png} &
\includegraphics[width=.1\linewidth]{MRTNet/rendering/selected/rec_shapenet/pc_0054.png}
\end{tabular}
\vspace{-12pt}
    \caption{\label{fig:gallery} 
    \small Test set shapes reconstructed by \mrvae trained on all categories of ShapeNet (using 80\%/20\% training/test split). \mrvae is able to reconstruct high-quality diverse shapes.}
    \vspace{-8pt}
\end{figure*}

\subsection{Unsupervised Learning of Point Clouds} \label{sec:exp_gen}
For unsupervised learning of point clouds, we train our \mrvae using the ShapeNet dataset~\cite{chang2015shapenet}. 
By default we compute $N=4K$ points for each shape using Poisson Disk sampling~\cite{Bowers:2010:PPD} to evenly disperse the points. 
Each point set is then spatially sorted using a \kdtree. 
Here we use the vanilla \kdtree where the splitting axes alternate between $x$, $y$, $z$ at each level of the tree. 
The spatially sorted points are used as input to train the \mrvae network (Section~\ref{mrt:method}). 
Similar to before, we also train a baseline model that follows the same network but replacing multiresolution convolutions with single-scale 1D convolutions
in both encoder and decoder. 
As Figure~\ref{fig:chairs-comp} shows, the shapes generated by the \mrvae trained on chairs are of considerably higher quality than those generated by the baseline model. 

We also performed multiple-category shape generation by training \mrvae on $80\%$ of the objects from ShapeNet dataset. 
The remaining models belong to our test split. 
Reconstructions of objects in the test split are included in Figure~\ref{fig:gallery}. 
Even when trained with a greater variety of shapes, the \mrvae is able to reconstruct high quality shapes from its embedding. 
This demonstrates that \mrvae is suitable for various inference tasks such as shape completion or point cloud reconstructions.

\para{Point ordering in the generated shapes.} 
A useful way to analyze shape generation is to see if the generated points have any consistent ordering across different shapes. 
This is an interesting question because as described previously, our \mrvae is trained using Chamfer Distance, 
a metric that's invariant to permutations of points. 
While the input to the network is all spatially sorted, the output is not restricted to any particular order and can in theory assume any arbitrary order. 
In practice, similar to the image-to-shape model, we observe that there is a consistent ordering of points in the generated shapes, as shown in Figure~\ref{fig:corresp}. 
Specifically, we picked three index ranges from one example chair, one at the top, one on the side, and one close to the bottom, 
then we color coded points in each shape that fall into these three index ranges. 
In the figure we can see clearly that they fall into approximately corresponding regions on each chair shape.

\begin{figure*}[t]
\centering
\setlength{\tabcolsep}{0pt}
\begin{tabular}{cccccccccccc}
\includegraphics[width=.083\linewidth]{MRTNet/rendering/selected/corresp/m1.png} &
\includegraphics[width=.083\linewidth]{MRTNet/rendering/selected/corresp/m2.png} &
\includegraphics[width=.083\linewidth]{MRTNet/rendering/selected/corresp/m3.png} &
\includegraphics[width=.083\linewidth]{MRTNet/rendering/selected/corresp/m4.png} &
\includegraphics[width=.083\linewidth]{MRTNet/rendering/selected/corresp/m5.png} &
\includegraphics[width=.083\linewidth]{MRTNet/rendering/selected/corresp/m6.png} &
\includegraphics[width=.083\linewidth]{MRTNet/rendering/selected/corresp/m7.png} &
\includegraphics[width=.083\linewidth]{MRTNet/rendering/selected/corresp/m8.png} &
\includegraphics[width=.083\linewidth]{MRTNet/rendering/selected/corresp/m9.png} &
\includegraphics[width=.083\linewidth]{MRTNet/rendering/selected/corresp/m10.png} &
\includegraphics[width=.083\linewidth]{MRTNet/rendering/selected/corresp/m11.png} &
\includegraphics[width=.083\linewidth]{MRTNet/rendering/selected/corresp/m12.png} 
\end{tabular}
\vspace{-8pt}
    \caption{\small \label{fig:corresp} Point correspondences among different shapes generated by \mrvae. 
	We picked three index ranges (indicated by three colors) from one example chair, and then color coded points in every shape that fall into these three ranges. 
	The images clearly show that the network learned to generate shapes with consistent point ordering.
    }
    \vspace{-4pt}
\end{figure*}

\begin{figure*}[t]
\centering
\setlength{\tabcolsep}{0pt}
\begin{tabular}{cccccccccc}
\includegraphics[width=.1\linewidth]{MRTNet/rendering/selected/interp1/pc_0000.png} &
\includegraphics[width=.1\linewidth]{MRTNet/rendering/selected/interp1/pc_0004.png} &
\includegraphics[width=.1\linewidth]{MRTNet/rendering/selected/interp1/pc_0006.png} &
\includegraphics[width=.1\linewidth]{MRTNet/rendering/selected/interp1/pc_0008.png} &
\includegraphics[width=.1\linewidth]{MRTNet/rendering/selected/interp1/pc_0009.png} &
\includegraphics[width=.1\linewidth]{MRTNet/rendering/selected/interp1/pc_0049.png} &
\includegraphics[width=.1\linewidth]{MRTNet/rendering/selected/interp1/pc_0058.png} &
\includegraphics[width=.1\linewidth]{MRTNet/rendering/selected/interp1/pc_0070.png} &
\includegraphics[width=.1\linewidth]{MRTNet/rendering/selected/interp1/pc_0080.png} &
\includegraphics[width=.1\linewidth]{MRTNet/rendering/selected/interp1/pc_0099.png} \\
\includegraphics[width=.1\linewidth]{MRTNet/rendering/selected/interp2/pc_0000.png} &
\includegraphics[width=.1\linewidth]{MRTNet/rendering/selected/interp2/pc_0004.png} &
\includegraphics[width=.1\linewidth]{MRTNet/rendering/selected/interp2/pc_0007.png} &
\includegraphics[width=.1\linewidth]{MRTNet/rendering/selected/interp2/pc_0009.png} &
\includegraphics[width=.1\linewidth]{MRTNet/rendering/selected/interp2/pc_0010.png} &
\includegraphics[width=.1\linewidth]{MRTNet/rendering/selected/interp2/pc_0011.png} &
\includegraphics[width=.1\linewidth]{MRTNet/rendering/selected/interp2/pc_0012.png} &
\includegraphics[width=.1\linewidth]{MRTNet/rendering/selected/interp2/pc_0014.png} &
\includegraphics[width=.1\linewidth]{MRTNet/rendering/selected/interp2/pc_0017.png} &
\includegraphics[width=.1\linewidth]{MRTNet/rendering/selected/interp2/pc_0019.png} 
\end{tabular}
    \vspace{-8pt}
    \caption{\small \label{fig:interp} Shape interpolation results. For each example, we obtain the encodings $\encoding$ of the starting shape and ending shape, then linearly interpolate the encodings and use the decoder to generate output shapes from the interpolated $\encoding$. Results show plausible interpolated shapes.
    }
    \vspace{-16pt}   
\end{figure*}

\para{Shape interpolation.} Another common test is shape interpolation: pick two encodings (either randomly sampled, or generated by the encoder for two input shapes), linearly interpolate them and use the decoder to generate the output shape. 
Figure~\ref{fig:interp} shows two sets of interpolation results of chairs from the ShapeNet dataset.


\para{Unsupervised classification.} A typical way of assessing the quality of representations learned in a unsupervised setting is to use them as features for classification. 
To do so, we take the \mrvae model trained with all ShapeNet objects, and use its features to classify ModelNet40~\cite{wu20153d} objects. 
Our classifier is a single linear layer, where the input is a set of features gathered from the first three layers of the \mrvae encoder. 
The features are constructed this way: we apply a pooling operation of size 128, 64 and 32 respectively on these three layers; 
then at each layer upsample the two smaller resolutions of features to the higher resolution such that all three resolutions have the same size. 
Finally, we concatenate all those features and pass them through a linear layer to get the final classification. 
It is important to notice that we did not perform any fine-tuning: the only learned parameters are those from the single linear layer. 
We used an Adam optimizer with learning rate $10^{-3}$ and $\beta=0.9$. 
The learning rate is decayed by dividing it by 2 every 5 epochs. 
Using this approach, we obtained an accuracy of $86.34\%$ on the ModelNet40 classification benchmark, as shown in Table~\ref{tab:class}(c). 
This result is considerably higher compared to similar features extracted from unsupervised learning in other autoencoders.
This shows that the representations learned by our \mrvae is more effective at capturing and linearizing the latent space of training shapes.


\vspace{-8pt}

\subsection{Discussions} \label{sec:discussions}
\para{Robustness to transformations.} Kd-trees are naturally invariant to point jittering as long as it's small enough so as to not alter the shape topology. 
Our approach is invariant to translations and uniform scaling as the models are re-centered at the origin and resized to fit in the unit cube. On the other hand, kd-trees are not invariant to rotations. 
This can be mitigated by using practices like pooling over different rotations (e.g. MVCNN) or branches that perform pose prediction and transformation (e.g. PointNet). 
However, we notice that simply having unaligned training data was enough to account for rotations in the classification task, and the ModelNet40 dataset contains plenty of unaligned shapes. 
Moreover, since the KDNet~\cite{Klokov_2017_ICCV} also employs a kd-tree spatial data structure, the discussions there about transformations also apply to our method.

\para{Computation time.} Building a kd-tree of $N$ points takes $O(N \log N)$ time, where $N=2^{10}$ for 1K points. 
While PointNet does not require this step, it's also more than 2.0\% worse in the classification task. 
The time to run a forward pass for classification is as follows: PointNet takes 25.3ms, while MRTNet takes 8.0ms on a TITAN GTX1080, both with batch size of 8.
Kd-tree building is also much faster than rendering a shape multiple times like in MVCNN~\cite{mvcnn} or voxelizing it~\cite{Riegler2017CVPR}. 
Using 16 different test-time augmentations does not have significant impact in computational time, as the 16 versions are classified in the same batch. 
This number of test-time augmentations is comparable to other approaches, e.g. 10 in~\cite{Klokov_2017_ICCV}, 80 in~\cite{mvcnn}, and 12 in~\cite{ocnn} and~\cite{pointnet}.


\section{Conclusion}

We presented a manifold prior induced by deep neural networks.
Our experiments show that the prior can be effectively used for a variety of manifold reconstruction
tasks: denoising, interpolation and single-view reconstruction.
Besides, we analyzed the influence of the architecture in the characteristics of the prior
by posing the models as GP.
In conjunction to the prior induced by deep networks, we showed that using a stretch regularization procedure enables
better manifold approximation and improves the quality of the generated meshes, reducing large deformations and overlaps
between different parameterizations.





\clearpage
{\small
\bibliographystyle{ieee}
\bibliography{egbib,mrnet}
}
\end{document}
