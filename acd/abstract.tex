%\rui{It may be better to start the abstract with some words about label-efficient learning, since that's the driving application. Right now it sounds like ACD is the driving application.}
The problems of shape classification and part segmentation from 3D
point clouds have garnered increasing attention in the last few
years. %~\cite{pointnet,qi2017pointnetpp,mrt18,su18splatnet}.  
But both of these problems suffer from relatively small training sets, creating the need for statistically efficient methods to learn 3D shape representations. In this work, 
we investigate the use of Approximate Convex Decompositions (ACD) as a self-supervisory signal for
label-efficient learning of point cloud representations.
Decomposing a 3D shape into simpler constituent parts or primitives is a fundamental problem in geometrical shape processing. 
There has been extensive work on such decompositions, where the criterion for simplicity of a constituent shape 
is often defined in terms of convexity for solid primitives.
% \elm{The following two sentences do not explain what supervision ACD provides. I think you want to say something like ``ACD provides a good initial solution for segmenting point clouds that, when used as approximate ground truth, provides excellent self-supervision for training a network to do segmentation. }In this paper, we posit that such a shape decomposition task is able to provide a useful self-supervisory signal to current deep neural networks used for various discriminative tasks on 3D point cloud data, such as part segmentation and shape classification. Specifically, we show that the well-known Approximate Convex Decomposition (ACD) of a point cloud serves as excellent self-supervision, allowing a model to learn from large amounts of unlabeled point cloud data. 
In this paper, we show that using the results of ACD to approximate a ground truth segmentation provides excellent self-supervision for learning 3D point cloud representations that are highly effective on downstream tasks. 
We report improvements over the state-of-the-art in unsupervised representation learning on the ModelNet40 
shape classification dataset and significant gains in few-shot part segmentation on the ShapeNetPart dataset.
